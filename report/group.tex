% compile: $ pdflatex group.tex

\documentclass[a4paper, notoc]{tufte-handout}

\title{Human Computer Interaction\\ Coursework 2 Report}

\author{Group 35}

%\date{1st January 1970} % without \date command, current date is supplied

%\geometry{showframe} % display margins for debugging page layout
\usepackage{appendix}
\usepackage{graphicx} % allow embedded images
\setkeys{Gin}{width=\linewidth,totalheight=\textheight,keepaspectratio}
\graphicspath{{graphics/}} % set of paths to search for images
\usepackage{amsmath}  % extended mathematics
\usepackage{booktabs} % book-quality tables
\usepackage{units}    % non-stacked fractions and better unit spacing
\usepackage{multicol} % multiple column layout facilities
\usepackage{lipsum}   % filler text
\usepackage{fancyvrb} % extended verbatim environments\
\fvset{fontsize=\normalsize}% default font size for fancy-verbatim environments

\let\origdescription\description
\renewenvironment{description}{
  \setlength{\leftmargini}{1.5em}
  \origdescription
  \setlength{\itemindent}{-1.5em}
  \setlength{\labelsep}{\textwidth}
}
{\endlist}


\begin{document}
\maketitle % this prints the handout title, author, and date
\vspace{1em}
\noindent
\begin{tabular}{l r}
  Ben Shaw       & s1338564\\
  Chris Campbell & s1334028\\
  Karel Kuzmiak  & s1334628\\
  Angus Pearson  & s1311631\\
\end{tabular}

%\tableofcontents
%\newpage

%% REQUIREMENTS %%

\section{Topics}

% The topic(s) you decided to cover in the website. What kind of information 
% you were trying to convey. 

\section{Design Motivation}

% Reasoning behind why the website is usable. This section will likely include 
% the outcome of any evaluations you conducted. Unlike the last coursework I do 
% not expect you to do a full formal analysis and description. A few sentences 
% on the method you used is fine. The key is providing reasoning on either why 
% the site is usable, or what you changed to make it so.

\section{Reflection}

% Reflection paragraph: what you have learned from completing this coursework. 
% What worked, what didn't, and if you could go back and do this coursework 
% again what might you do differently the next time.


\section{Tools and Templates}

%% REQUIREMENTS %%
% A list of any tools or templates you used to construct the website.

\begin{description}

\item[Jekyll]
Static site generator (used by GitHub pages) written in Ruby. Supports Markdown for 
content markup and the \textit{SASS} CSS pre-processor.
\\
\href{http://jekyllrb.com/}{http://jekyllrb.com/}

\item[BootStrap]
Web UI framework (CSS, JavaScript) freely available, created by Twitter.
\\
\href{http://getbootstrap.com/}{http://getbootstrap.com/}

\item[FontAwesome]
Web Icon set, freely available.
\\
\href{http://fontawesome.io/}{http://fontawesome.io/}


\item[University of Edinburgh Style Guide]
Fonts -- Serif:- \textit{Crimson Text}, Sans-Serif: \textit{Source Sans Pro} both 
available as \href{https://fonts.google.com/}{Google Web Fonts}.
\\
Colours are taken from 
\href{http://www.ed.ac.uk/communications-marketing/resources}{\textit{Brand Guidelines}} 
and associated pages.
\\
\href{http://fontawesome.io/}{http://fontawesome.io/}


\end{description}




\section*{Mark Allocation}

%% REQUIREMENTS %%

% How the group would like me to allocate marks. Two options: 1) everybody gets 
% the same mark, 2) a clear list of who is responsible for which part of the 
% website or evaluation. In the case of #2 10 points will be marked group wide 
% for the general design of the website and the remaining 40 points will be marked 
% based on the individual portion of the work.

1) Everybody gets the same mark


%       ^v^v^v^v^v^v^v^v^v^v^v^v^v^v^v^v^v^v^v^v^v^v^v^v^v^v^v^v^v^v^v^v^v^v^v^


%\bibliography{group}
%\bibliographystyle{plainnat}

\end{document}

